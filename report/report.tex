\documentclass[12pt]{article}

\usepackage[utf8]{inputenc}
\usepackage{listings}
\usepackage{graphicx}
\usepackage{float}
\usepackage{geometry}
\usepackage{authblk}
\usepackage{setspace}

\newcommand*{\TitleFont}{
  \usefont{\encodingdefault}{\rmdefault}{b}{n}
  \fontsize{30}{40}
  \selectfont}

\usepackage{parskip}
\setlength{\parskip}{1.0\baselineskip plus2pt minus2pt}
%% \setlength{\baselineskip}{1cm}

\addtolength{\topmargin}{-50pt}
\addtolength{\textheight}{130pt}
\addtolength{\textwidth}{95pt}
\addtolength{\oddsidemargin}{-45pt}

\title{\TitleFont{Computação Gráfica}}
\author{David Gomes (2013136061) \and \vspace{-0.1cm} André Baptista (2013136742)}
\date{}

\begin{document}
\maketitle

\section*{Introdução}
O nosso projeto de Computação Gráfica consiste num ambiente 3D com várias bolas a movimentar-se
dentro de um cubo coberto, na sua base, por água.

\begin{figure}[H]
  \centering
  \includegraphics[width=0.75\textwidth]{screenshot}
\end{figure}

O objetivo deste relatório é apresentar algumas das \textit{features} do nosso projeto e explicar sucintamente a sua implementação.

\pagebreak

\subsection*{\textit{Object Loader}}
Para facilitar o desenvolvimento do projeto criamos um \textit{loader} de objetos que desenha modelos
a partir de ficheiros \texttt{.obj}. Com este mecanismo associado à classe \texttt{Object}, carregamos o modelo da ilha assim como o do cubo.
\subsection*{Sol}
\subsection*{Reflexão}
\subsection*{Água}
\subsection*{\textit{Skybox}}
\subsection*{\textit{Lens Flare}}
\subsection*{Colisões}
Vários tipos de colisões simples são usadas no nosso projeto, seja para detetar colisão entre duas esferas ou entre esferas e planos.

\subsection*{Explosões (partículas)}
Implementamos um efeito de partículas que gera centenas de partículas (implementadas com pequenos cubos tridimensionais de OpenGL) para simular uma explosão fogosa quando duas bolas colidem.

\subsection*{Lançamento de Bolas}
Ao carregarmos com o rato esquerdo no ecrã será "atirada" para a cena uma nova bola com a direção
do observador que, tal como qualquer outra bola, interage com as restantes (podendo portanto colidir com qualquer coisa).

\end{document}